\begin{hcarentry}{logfloat}
\report{Wren G. Romano}%01/14
\status{provisional?}
\release{0.14}
\portability{GHC 7.8}% does Hugs Sept2006 still work?
\makeheader

The logfloat library provides a type for storing numbers in the log-domain. This is primarily useful for avoiding underflow when multiplying many small numbers in probabilistic models.

It also includes support for dealing with IEEE-754 floating point numbers (more) correctly, including: a class for types with representations for transfinite values, a class for partially ordered types, efficient and correct conversion from \texttt{Real} to \texttt{Fractional}, and bug fixes for Hugs' Prelude.

\FuturePlans
Add a signed variant so negative numbers can also be projected into the log-domain.

\FurtherReading
\begin{compactitem}
\item Official source and documentation available on Hackage
\item The development branch is available from \url{http://community.haskell.org/~wren/}
\end{compactitem}
\end{hcarentry}
